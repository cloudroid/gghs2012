% Latex source for "Rapid 3D inversion of gravity and gravity gradient data
% to test geologic hypotheses" by Leonardo Uieda and Valeria C F Barbosa

\RequirePackage{fix-cm}

%\documentclass[twocolumn,final]{svjour3}
\documentclass[twocolumn,draft]{svjour3}
%\documentclass[twocolumn,referee]{svjour3}

\smartqed  % flush right qed marks, e.g. at end of proof

\usepackage{graphicx}
\usepackage{amsmath}
\usepackage[round,sort]{natbib}

\journalname{International Association of Geodesy Symposia}



\begin{document}

\title{
    Rapid 3D inversion
    of gravity and gravity gradient data
    to test geologic hypotheses
}
\author{Leonardo Uieda \and Val\'eria C. F. Barbosa}

%\subtitle{Do you have a subtitle?\\ If so, write it here}
%\titlerunning{Short form of title}        % if too long for running head
%\authorrunning{Short form of author list} % if too long for running head

\institute{
    L. Uieda \and V.C.F. Barbosa \at
        Observat\'orio Nacional,
        Rua General Jos\'e Cristino 77,
        20921-400 Rio de Janeiro - RJ, Brazil.
        \email{leouieda@gmail.com; valcris@on.br}
}

\date{Received:  / Accepted: }

\maketitle

\begin{abstract}
% Note: Max 250 words
Forward modeling of potential fields
is a useful way to incorporate the interpreter's knowledge
about the geology of the interpretation area into the model.
However, this can be a very tedious task.
This is specially true when modeling in 3D
and trying to fit multiple components,
e.g., in gravity gradiometry.
The interpreter is required
to simultaneously supervise the data fit
and the construction of geologically realistic 3D bodies.
This problem is partially solved
by methods of geophysical inversion,
which automatically fit the data.
Conversely,
inverse problems introduce other challenges of their own.
Most geophysical inverse problems are ill-posed
because their solutions are neither unique nor stable.
Thus, they require the introduction of prior information,
usually through regularizing functions.
Moreover, 3D inverse problems are very computationally expensive.
Recent developments in potential field inversion
have proposed different regularizing functions
to transform the ill-posed problem into a well-posed one.
Also, several techniques,
like data compression and parallel computation,
have been applied to overcome the computational complexity.
We call attention to
the method of potential field inversion
by planting anomalous densities.
This method uses an iterative algorithm
to automatically grow the anomalous bodies
around user-specified prismatic elements called ``seeds'',
which have fixed density contrasts and positions.
These seeds provide a first estimate
of the skeletal outlines of the presumed anomalous bodies.
Then, the inversion iteratively concentrates mass around this ``skeleton''
in a way that both fits the observed data and yields compact bodies.
Therefore,
the interpreter can easily impose prior information
on the inversion through the seeds.
The interpreter needs only to supply a few seeds
that specify the sources' skeleton,
eliminating the exhaustive task
of specifying the complete geometry of multiple sources.
Moreover, the interpreter is liberated
from the time-consuming procedure
of yielding a reasonable fit to the data.
Due to its high computational efficiency,
the method of planting anomalous densities
can be used to quickly test geologic hypothesis
of different locations and density contrasts
for presumed sources.
To test a hypothesis,
one would choose
the locations and density contrasts of the seeds accordingly
and verify if the inversion result
is able to fit the observed data.
If it is not able,
then the hypothesis can be rejected
and a new one can be formulated and tested.
Otherwise, there is no reason to reject the hypothesis
on the basis of the geophysical data.
Thus, the method can be viewed as a an enhanced forward modeling.
The method of planting anomalous densities
can be used with both gravity and gravity gradient data.
This makes it an ideal tool
to interpret compact geologic bodies
using the new generation GOCE data.
We present  applications to synthetic and real data
that illustrate the usefulness of our method.

\keywords{
    Gravity inversion \and
    Gravity gradiometry \and
    Modeling \and
    3D \and
    GOCE
}
\end{abstract}

% INTRODUCTION
%%%%%%%%%%%%%%%%%%%%%%%%%%%%%%%%%%%%%%%%%%%%%%%%%%%%%%%%%%%%%%%%%%%%%%%%%%%%%%%%
\section{Introduction}
\label{intro}

Forward modeling of potential fields
is a useful way to incorporate the interpreter's knowledge
about the geology of the interpretation area into the model.
However, this can be a very tedious task.
This is specially true when modeling in 3D
and trying to fit multiple components,
e.g., in gravity gradiometry.
The interpreter is required
to simultaneously supervise the data fit
and the construction of geologically realistic 3D bodies.
This problem is partially solved
by methods of geophysical inversion,
which automatically fit the data.
Conversely,
inverse problems introduce other challenges of their own.
Most geophysical inverse problems are ill-posed
because their solutions are neither unique nor stable.
Thus, they require the introduction of prior information,
usually through regularizing functions.
Moreover, 3D inverse problems are very computationally expensive.
Recent developments in potential field inversion
have proposed different regularizing functions
to transform the ill-posed problem into a well-posed one.
Also, several techniques,
like data compression and parallel computation,
have been applied to overcome the computational complexity.
We call attention to
the method of potential field inversion
by planting anomalous densities.
This method uses an iterative algorithm
to automatically grow the anomalous bodies
around user-specified prismatic elements called ``seeds'',
which have fixed density contrasts and positions.
These seeds provide a first estimate
of the skeletal outlines of the presumed anomalous bodies.
Then, the inversion iteratively concentrates mass around this ``skeleton''
in a way that both fits the observed data and yields compact bodies.
Therefore,
the interpreter can easily impose prior information
on the inversion through the seeds.
The interpreter needs only to supply a few seeds
that specify the sources' skeleton,
eliminating the exhaustive task
of specifying the complete geometry of multiple sources.
Moreover, the interpreter is liberated
from the time-consuming procedure
of yielding a reasonable fit to the data.
Due to its high computational efficiency,
the method of planting anomalous densities
can be used to quickly test geologic hypothesis
of different locations and density contrasts
for presumed sources.
To test a hypothesis,
one would choose
the locations and density contrasts of the seeds accordingly
and verify if the inversion result
is able to fit the observed data.
If it is not able,
then the hypothesis can be rejected
and a new one can be formulated and tested.
Otherwise, there is no reason to reject the hypothesis
on the basis of the geophysical data.
Thus, the method can be viewed as a an enhanced forward modeling.
The method of planting anomalous densities
can be used with both gravity and gravity gradient data.
This makes it an ideal tool
to interpret compact geologic bodies
using the new generation GOCE data.
We present  applications to synthetic and real data
that illustrate the usefulness of our method.

% METHODOLOGY
%%%%%%%%%%%%%%%%%%%%%%%%%%%%%%%%%%%%%%%%%%%%%%%%%%%%%%%%%%%%%%%%%%%%%%%%%%%%%%%%
\section{Methodology}
\label{method}

Text with citations \cite{RefB} and \cite{RefJ}.

\section{Example application}
\label{application}
as required. Don't forget to give each section
and subsection a unique label (see Sect.~\ref{method}).

\begin{equation}
a^2+b^2=c^2
\end{equation}

% For one-column wide figures use
\begin{figure}
% Use the relevant command to insert your figure file.
% For example, with the graphicx package use
  %\includegraphics{example.eps}
% figure caption is below the figure
\caption{Please write your figure caption here}
\label{fig:1}       % Give a unique label
\end{figure}

% For two-column wide figures use
\begin{figure*}
% Use the relevant command to insert your figure file.
% For example, with the graphicx package use
  %\includegraphics[width=0.75\textwidth]{example.eps}
% figure caption is below the figure
\caption{Please write your figure caption here}
\label{fig:2}       % Give a unique label
\end{figure*}

% For tables use
\begin{table}
% table caption is above the table
\caption{Please write your table caption here}
\label{tab:1}       % Give a unique label
% For LaTeX tables use
\begin{tabular}{lll}
\hline\noalign{\smallskip}
first & second & third  \\
\noalign{\smallskip}\hline\noalign{\smallskip}
number & number & number \\
number & number & number \\
\noalign{\smallskip}\hline
\end{tabular}
\end{table}

% CONCLUSIONS
%%%%%%%%%%%%%%%%%%%%%%%%%%%%%%%%%%%%%%%%%%%%%%%%%%%%%%%%%%%%%%%%%%%%%%%%%%%%%%%%
\section{Conclusions}
\label{conclusions}

% ACKNOWLEDGEMENTS
%%%%%%%%%%%%%%%%%%%%%%%%%%%%%%%%%%%%%%%%%%%%%%%%%%%%%%%%%%%%%%%%%%%%%%%%%%%%%%%%
\begin{acknowledgements}
If you'd like to thank anyone, place your comments here
and remove the percent signs.
\end{acknowledgements}


% REFERENCES
%%%%%%%%%%%%%%%%%%%%%%%%%%%%%%%%%%%%%%%%%%%%%%%%%%%%%%%%%%%%%%%%%%%%%%%%%%%%%%%%
\begin{thebibliography}{}

\bibitem[Cartwright, 2007]{RefB}
Cartwright J (2007) Big stars have weather too. IOP Publishing PhysicsWeb.
http://physicsweb.org/articles/news/11/6/16/1. Accessed 26 June 2007

\bibitem[Gamelin et~al., 2009]{RefJ}
Gamelin FX, Baquet G, Berthoin S, Thevenet D, Nourry C, Nottin S, Bosquet L
(2009) Effect of high intensity intermittent training on heart rate variability
in prepubescent children. Eur J Appl Physiol 105:731-738.
doi: 10.1007/s00421-008-0955-8


\end{thebibliography}

\end{document}

